In 2020, 30 students took the exam, compared to TODO students in 2019 (2018).
The students solved the exam assignments on paper and submitted photographs of their
solutions.

The types of exam problems were the same as in previous years.
However, contrary to previous years, different students had different exam
assignments, to minimise opportunity for collaboration between students.
%
While building the pipeline described in this paper required much more work
than creating just one exam sheet, our approach was more efficient than
it would be to create 30~different exam sheets manually. Additionally,
our approach guaranteed that the exam problems were
unique, yet required comparable effort to solve.
Also, reusing our pipeline in the future requires only minimal changes.

Further, the types of the problems in our exam are not trivial to grade, since
the solutions require applying complicated reasoning algorithms on paper, and
the grade has to take into account the whole process, not just the result.
However, the use of templates of Section~\ref{sec:smtqf}
made the grading fairly similar to grading multiple
solutions of the same problem by providing a clear pattern to follow.
%
%  4 -- underlying argument was independent of term structure,
%       so that was easy to check,
%       what varied per student was the substitution/mgu
This observation extends to %problem 4,
the problem on non-ground superposition (Subsection~\ref{sec:fo}),
because the argument required in the solution does not depend majorly on the generated parts,
even though we did not use an explicit template.
% The same holds for problem 4, where we do not use an explicit template
% but the argument does not depend majorly on the generated parts.
%
%  1 -- more work to check as all steps may vary depending on the input;
%       (to help here, one could automate also the solution generation;
%       but: at some points there are multiple correct possibilities which affect the way forward,
%       order is not fixed,
%       and: if the student makes a mistake, we still give points for subsequent work
%           if the reasoning is correct (even if the result is not).
The situation is different for %problem 1, however.
the problem on boolean satisfiability (Subsection~\ref{sec:sat}).
There, the solution varies greatly with the input formula,
and grading a different instance requires mentally stepping through the problem again.
% It might be interesting future work to also automatically generate fully worked solutions
One might suggest to also generate fully worked solutions to this problem,
however it is not immediately clear that this would be helpful:
at various points, the students may choose among multiple correct possibilities,
each of which leads to differences in subsequent parts of the solution.

The average exam score was 79.9\%, compared to TODO in 2019 (2018).
\todo{If the scores are comparable, say something like "Thus, we believe
that the online setup was sufficiently similar to the in-person examination."}

Finally, 8 students filled out a feedback survey for the course. All of them reported
high levels of satisfaction with the course, with one student explicitly praising the
online exam format.
