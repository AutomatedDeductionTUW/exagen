% vim:ft=tex:
%
\RequirePackage[l2tabu, orthodox]{nag}
\documentclass[12pt]{llncs}
\pagestyle{headings}


%%%%%%%%%%%%%%%%%%%%%%%%%%%%%%%%%%%%%%%
% FONTS
%%%%%%%%%%%%%%%%%%%%%%%%%%%%%%%%%%%%%%%

\usepackage{lmodern}        % Improved version of the original Computer Modern font
\usepackage{stmaryrd}       % For \llbracket and \rrbracket
% Define font from package 'stmaryrd' for arbitrary size,
% see https://tex.stackexchange.com/a/128801
\DeclareFontFamily{U}{stmry}{}
\DeclareFontShape{U}{stmry}{m}{n}
{
    <-5.5>    stmary5
    <5.5-6.5> stmary6
    <6.5-7.5> stmary7
    <7.5-8.5> stmary8
    <8.5-9.5> stmary9
    <9.5->    stmary10
}{}

\usepackage[T1]{fontenc}    % Determines font encoding of the output. Font packages have to be included before this line.
\usepackage[utf8]{inputenc} % Determines encoding of the input. All input files have to use UTF8 encoding.
\usepackage[english]{babel}


%%%%%%%%%%%%%%%%%%%%%%%%%%%%%%%%%%%%%%%
% PACKAGES
%%%%%%%%%%%%%%%%%%%%%%%%%%%%%%%%%%%%%%%

\usepackage{amsmath}
\usepackage{amssymb}
% \usepackage{thmtools}
% \usepackage{mathtools}
% \usepackage{prftree}
% \usepackage{cancel}
% \usepackage{keycommand}
\usepackage{listings}
% \usepackage{lstautogobble}
\usepackage{microtype}  % Small-scale typographic enhancements.
\usepackage[shortlabels]{enumitem}
\usepackage[hidelinks]{hyperref}  % optional argument [hidelinks] hides colored boxes around clickable areas
\usepackage{prftree}
\usepackage{mathtools}
\usepackage{booktabs}   % Improves the typesettings of tables.
\usepackage{tabularx}
\usepackage{multirow}   % Allows table elements to span several rows.
% \usepackage{pdfpages}  % to include pages from existing pdf files with \includepdf[pages={1,2}]{./path/to/file.pdf}
\usepackage{xcolor}
\usepackage{todonotes}  % Add optional argument [disable] to hide TODOs, or [obeyFinal] to hide only in final mode

\usepackage{tikz}
\usetikzlibrary{arrows, arrows.meta, backgrounds, calc, decorations.markings, positioning, shapes.geometric}


%%%%%%%%%%%%%%%%%%%%%%%%%%%%%%%%%%%%%%%
% SETTINGS
%%%%%%%%%%%%%%%%%%%%%%%%%%%%%%%%%%%%%%%

% \setlist{nolistsep}  % less vertical spacing for itemize/enumerate environments


%%%%%%%%%%%%%%%%%%%%%%%%%%%%%%%%%%%%%%%
% MACROS
%%%%%%%%%%%%%%%%%%%%%%%%%%%%%%%%%%%%%%%

\makeatletter
% \def\UrlFont{\rmfamily}
% \def\orcidID#1{\smash{\href{http://orcid.org/#1}{\protect\raisebox{-1.25pt}{\protect\includegraphics{orcid_color.eps}}}}}
% The \smash breaks the hyperlink... the clickable area becomes tiny.
\def\orcidID#1{{\href{http://orcid.org/#1}{\protect\raisebox{-1.25pt}{\protect\includegraphics{ORCID_Color.eps}}}}}
\makeatother

\newcommand{\todoi}[1]{\todo[inline,caption={}]{#1}}    % caption={} allows itemize etc. inside todoi, see https://tex.stackexchange.com/a/54068

% This is the provability analogue to \models (note that \vdash is smaller)
\DeclareRobustCommand\proves{\mathrel{|}\joinrel\mkern-.5mu\mathrel{-}}

\newcommand{\limpl}{\rightarrow}    % logical implication
\newcommand{\Limpl}{\Rightarrow}    % logical implication (variant)
\newcommand{\liff}{\leftrightarrow} % logical equivalence
\newcommand{\Liff}{\Leftrightarrow} % logical equivalence (variant)

\newcommand{\Land}{\bigwedge}
\newcommand{\Lor}{\bigvee}

\newcommand{\union}{\cup}
\newcommand{\Union}{\bigcup}
\newcommand{\intersect}{\cap}
\newcommand{\Intersect}{\bigcap}

\newcommand{\eql}{\simeq}
\newcommand{\neql}{\not\eql}

% \newcommand{\nat}{\mathbb{N}}
% \newcommand{\int}{\mathbb{Z}}

\newcommand{\naf}{{\sim}}           % negation as failure, default negation
\newcommand{\caF}{\mathcal{F}}
\newcommand{\caM}{\mathcal{M}}
\newcommand{\caT}{\mathcal{T}}

\newcommand{\vampire}{\textsc{Vampire}}


%%%%%%%%%%%%%%%%%%%%%%%%%%%%%%%%%%%%%%%
% METADATA
%%%%%%%%%%%%%%%%%%%%%%%%%%%%%%%%%%%%%%%

\title{Automated Reasoning in Generating Exam Sheets for Automated Deduction}
\author{Petra Hozzov\'a\and
Laura Kov\'acs \and
Jakob Rath}

\institute{
    TU Wien, Austria
    % \and
    % Princeton University, Princeton NJ 08544, USA \and
    % Springer Heidelberg, Tiergartenstr. 17, 69121 Heidelberg, Germany
    % \email{lncs@springer.com}\\
    % \url{http://www.springer.com/gp/computer-science/lncs} \and
    % ABC Institute, Rupert-Karls-University Heidelberg, Heidelberg, Germany\\
    % \email{\{abc,lncs\}@uni-heidelberg.de}
}

% Set PDF document properties
\hypersetup{
    pdfpagemode     = UseNone,                  % Don't show bookmarks when opening the pdf file, see also https://latex.org/forum/viewtopic.php?p=70568#p70568
    pdfpagelayout   = OneColumn,                % How the document is shown in PDF viewers (optional). Values: SinglePage, OneColumn, TwoColumnLeft, TwoColumnRight, TwoPageLeft, TwoPageRight
    % pdfauthor       = {\authorname},          % The author's name in the document properties (optional).
    % pdftitle        = {\thesistitle},         % The document's title in the document properties (optional).
    % pdfsubject      = {subject},              % The document's subject in the document properties (optional).
    % pdfkeywords     = {keyword, another keyword}, % The document's keywords in the document properties (optional).
    colorlinks,
    % linkbordercolor = {Melon},                % The color of the borders of boxes around crosslinks (optional).
    % linkcolor={black},
    % citecolor={black},
    % urlcolor={black},
}

\lstset{%
    basicstyle=\ttfamily,%
    % keywordstyle=\bfseries\color{blue},%
    columns=flexible,%
    % numberstyle=\tiny,%
    % numbersep=5pt,%
    % numbers=left,%
}


%%%%%%%%%%%%%%%%%%%%%%%%%%%%%%%%%%%%%%%
\begin{document}
\maketitle





\begin{abstract}
Amid the COVID-19 pandemic, distance teaching became the default in world-wide higher education,
urging teachers and researchers to revise course materials into a
more accessible online content for a diverse audience. Probably one of the hardest challenges in this new form of education
came with online assessments of course performance, especially organizing and grading online written exams.
In this short paper we focus on our teaching experience during our master's level course ``Automated Deduction'' at TU Wien,
and report on the automated reasoning we developed for generating individual online exam sheets for students enrolled in the course.
The algorithmic and rigorous logical reasoning developed within our course calls for exam sheets focused on problem solving and deductive proofs; as such exam sheets using test grids are not a viable solution for written exams within our course. We believe that the toolchain of automated reasoning tools we have developed for holding online written exams could be beneficial not only for other distance learning platforms, but also to researchers in automated reasoning by providing our community with a large set of randomly generated benchmarks in SAT/SMT solving and first-order theorem proving.
\end{abstract}





\section{Motivation}

online exams due to pandemic.
we want to avoid collaboration between students during exam (or make it at least a bit harder),
so each student gets their own exam sheet.
etc…





\section{todo}

Challenge:
problem instances should be different but of similar difficulty to make sure the exam is fair to students.





\section{Method 1: Varying Templates and Fixed Patterns}


\subsection{SMT problem}

relatively easy: provided template, do small random perturbations that don't change the solution


\subsection{Ground superposition problem}

todo




\section{Method 2: Full Random Generation}

more sophisticated: full random generation, filter out "too hard"/"too easy" instances.
Note that we don't need very efficient implementation of filters since the instances are very small.
so we can use naive satisfiability tests or model counting.


\subsection{SAT problem}

describe filters, and why they were chosen (aim for a challenging problem, but still solvable by hand).

problems:
when restricting too much, the resulting formulas may end up boring.
e.g. SAT formula with exactly one model, or no model


\subsection{Superposition+Redundancy problem}

todo





% \bibliographystyle{splncs04}
% \bibliography{references}
% % \bibliography{/Users/jakob/Documents/Library/library.bib}

\end{document}
