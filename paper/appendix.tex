% \subsection{Implementation Details on Random Problem Generation}\label{appB:SAT}

% As we alluded to in Section~\ref{sec:sat},
% some issues may arise if the restrictions on the randomly generated formula are too strict:
% \begin{itemize}
%     \item
%         The sample space might be empty or very sparse.
%         In practice, it seems to the user as if the generator got stuck,
%         usually resulting in the process being killed by the user. 
%         For example,
%         consider the restriction on polarities of propositional
%         variables. % (Sect.~\ref{sec:sat}).
%         Combined with the other restrictions,
%         it is impossible to get a formula that contains atomic propositions
%         of purely positive and purely negative polarity at the same time.

%     \item
%         The second issue manifests less drastically but is perhaps more problematic:
%         the sample space may be too uniform,
%         leading to the generation of trivial and/or very similar
%         formulas. 
%         In particular, we encountered this problem
%         when we restricted the number of models to exactly one, or zero.
%         We note that there simply are not that many ways to rule out 8 interpretations
%         using only 7 connectives.
% \end{itemize}





% Regarding the filtering of generated formulas using the constraints discussed in Section~\ref{sec:satfo},
% we did not require very efficient algorithms since
% the formulas under consideration are very small.
% For example, for the restriction on the number of models we used a naive satisfiability test
% based on evaluating the formula under each possible interpretation.
% An advantage of this is that the addition of new filters/constraints is relatively easy.




% For the random problem generation setting of Section~\ref{sec:satfo},
% we applied design principles of the Haskell library
% QuickCheck~\cite{ClaessenHughes:2000:QuickCheck}.
% Using QuickCheck, randomly generated data can easily be defined in an embedded generator language.
% However, because of our many filtering criteria, we wanted the generator to additionally support backtracking.
% Furthermore, we were interested in determining the size of the filtered sample space.
% To this end,
% we created a simple typeclass \texttt{MonadChoose} in the style of the monad transformer library (mtl),
% with a single primitive operation \texttt{choose} for choosing an element from a list of possible choices:
% \lstset{
%     language=Haskell,
%     basicstyle=\small\ttfamily,
%     deletekeywords={MonadPlus,filter,length},
%     columns=fixed,
% }
% \begin{lstlisting}
% class MonadPlus m => MonadChoose m where
%   -- Choose element with uniform probability
%   choose :: [a] -> m a
% \end{lstlisting}

% Our generator implementations are generic over the monad, constrained by \texttt{MonadChoose}.
% The following listing shows (a slightly simplified) part of the inference generator
% discussed in Section~\ref{sec:fo}.
% \begin{lstlisting}
% genExamInference :: MonadChoose m => m Inference
% genExamInference = do
%   -- Define signature (partially omitted)
%   let vars = ["x", "y", "z"]
%   let opts = GenOptions{ vars = vars, ... }

%   -- Choose variables to appear in l1 and l2
%   v1 <- choose vars
%   v2 <- choose (filter (/= v1) vars)

%   -- Generate literals
%   -- l1: exactly one occurrence of v1
%   l1 <- mfilter ((==1) . length . toListOf variables)
%         $ genUninterpretedLiteral opts{ vars = [v1] }
%   -- l2: at least two occurrences of v2
%   l2 <- mfilter ((>=2) . length . toListOf variables)
%         $ genEqualityLiteral opts{ vars = [v2] }
%   -- l3: ground literal
%   l3 <- genUninterpretedLiteral opts{ vars = [] }

%   -- (rest omitted)
%   return inference

% genEqualityLiteral, genUninterpretedLiteral
%   :: MonadChoose m => GenOptions -> m Literal
% -- (literal generators omitted)
% \end{lstlisting}

% We used two concrete implementations to evaluate generators:
% \begin{enumerate}
%     \item
%         \texttt{RandomChoice}, a monad that implements \texttt{choose}
%         as uniform random choice with backtracking support.
%         Conceptually, this is like the standard list monad
%         where \texttt{choose} works like the regular monadic bind for lists
%         except that it shuffles the list with a random permutation first.
%         This evaluation method is used to generate random exams.
%     \item
%         The standard list monad to enumerate the sample space.
%         This second evaluation method helps verifying that the sample space is sufficiently large.
% \end{enumerate}

%\subsection{Weights and Precedences for Ground Superposition}\label{appB:FO}
%
%The weights and precedences used to generate the KBOs for the superposition problem
%from Section~\ref{sec:gs}
%are displayed in Table~\ref{tab:ground-sup}.
%The upper part of the table shows all weight and precedence combinations,
%denoted as $w_{i, I}, p_{i, I}$ for $i \in \{1, 2, 3, 4\}$
%and $I \in \{f, g\}$ (for convenience, the table contains both $w_{i, f}$
%and $w_{i, g}$, as well as $p_{i, f}$ and $p_{i, g}$ for all values of~$i$).
%The lower part of the table displays the values of $i_1, I_1; i_2, I_2$; and $i_3, I_3$,
%corresponding to the three weight and precedence combinations selected for
%each instance of the clause set.\todo{Make this easier to read?}
%
%\begin{table}
%\begin{center}
%\begin{tabular}{r@{\hskip 0.5em}c c c c@{\hskip 0.5em} |@{\hskip 0.5em} l@{\hskip 1em} ||@{\hskip 0.5em} r@{\hskip 0.5em} c c c c@{\hskip 0.5em} |@{\hskip 0.5em} l}
%  weight of: & $f$ & $g$ & $a$ & $b$ & precedence
%  & weight of: & $f$ & $g$ & $a$ & $b$ & precedence \\ \hline
%  $w_{1,f}:$  & 1   & 3   & 2   & 1   & $p_{1,f}: a \gg b \gg f \gg g$ &
%  $w_{1,g}:$  & 3   & 1   & 2   & 1   & $p_{1,g}: a \gg b \gg g \gg f$ \\ \hline
%  $w_{2,f}:$  & 0   & 3   & 2   & 1   & $p_{2,f}: f \gg a \gg g \gg b$ &
%  $w_{2,g}:$  & 3   & 0   & 2   & 1   & $p_{2,g}: g \gg a \gg f \gg b$ \\ \hline
%  $w_{3,f}:$  & 0   & 1   & 3   & 1   & $p_{3,f}: f \gg a \gg b \gg g$ &
%  $w_{3,g}:$  & 1   & 0   & 3   & 1   & $p_{3,g}: g \gg a \gg b \gg f$ \\ \hline
%  $w_{4,f}:$  & 1   & 2   & 3   & 1   & $p_{4,f}: g \gg f \gg a \gg b$ &
%  $w_{4,g}:$  & 2   & 1   & 3   & 1   & $p_{4,g}: f \gg g \gg a \gg b$ \\ \hline
%  % original table:
%  %$w_1(h_1) = w_1(b) = 1, w_1(a) = 2, w_1(h_2) = 3$ & \quad $p_1: a \gg b \gg h_1 \gg h_2$ \\
%  %$w_2(h_1) = 0, w_2(b) = 1, w_2(a) = 2, w_2(h_2) = 3$ & \quad $p_2: h_1 \gg a \gg h_2 \gg b$ \\
%  %$w_3(h_1) = 0, w_3(b) = w_3(h_2) = 1, w_3(a) = 3$ & \quad $p_3: h_1 \gg a \gg b \gg h_2$ \\
%  %$w_4(b) = w_4(h_1) = 1, w_4(h_2) = 2, w_4(a) = 3$ & \quad $p_4: h_2 \gg h_1 \gg a \gg b$ \\
%\end{tabular}
%
%\hspace*{0.5em}
%
%\begin{tabular}{l@{\hskip 1.05em} || c | c | c}
%\hline
%condition & $i_1, I_1$ & $i_2, I_2$ & $i_3, I_3$ \\ \hline
%$F \not = G$ and $X \not= Y$ & $1,E$ & $2, E$ & $3, E$ \\
%$F \not = G$ and $X = Y$ & $1, H$ & $2, E$ & $4, H$ \\
%$F = G$ and $X \not= Y$ & $1, H$ & $2, H$ & $3, E$
%\end{tabular}
%\hspace*{0.5em}
%\caption{Weights and precedences for the ground superposition problem.}
%\label{tab:ground-sup}
%\end{center}
%\end{table}
