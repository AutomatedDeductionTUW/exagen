We considered first-order formula templates in the combined,
quantifier-free 
theories of equality, arrays and linear integer arithmetic,
corresponding to the logic AUFLIA of
SMT-LIB~\cite{barrett2017smtlib}. We aimed at generating SMT formulas
over which reasoning in all three theories was needed, by exploiting
the DPLL(T) framework~\cite{Tinelli02} in combination with the
Nelson-Oppen decision procedure~\cite{Nelson79} (Problem 2 of Figure~\ref{fig:exam}).
%
%One of our goals for this problem was to ensure that a correct solution
%requires combined reasoning in the theories of arrays, equality, and integers.

With naive random generation, it might however happen
that, for example, array reasoning is actually not needed to derive
(un)satisfi\-ability of the generated SMT formula. We therefore constructed an
SMT formula template and randomly introduced small perturbations in this template,
so that the theory-specific reasoning in all generated SMT instances is
different while reasoning in all theories is necessary.
%
For doing so, we considered an SMT template with two constants of
integer sort and replaced an integer-sorted constant symbol $c$ by  integer-sorted terms $c+i$,
where $i \in \{-3,-2,\dots,3\}$ is chosen randomly. We flattened nested arithmetic terms such as $(c+i)+j$ to $c+k$,
where $i,j,k$ are integers and $k = i+j$. As a result, we
generated 49 different SMT problems; we show one such formula, together
with the corresponding reasoning tasks, in Problem~2 of Figure~\ref{fig:exam}.
