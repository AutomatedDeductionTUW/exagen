In this problem, a formula in the SMT-LIB logic AUFLIA was given,
and the task was to apply DPLL(T) with the Nelson-Oppen method
to decide the given formula.

One of our goals for this problem was to ensure that a correct solution
requires combined reasoning in the theories of arrays, equality, and integers.
With naive random generation it might well happen that,
e.g., array operations appear in the formula,
but no actual array reasoning is required to prove unsatisfiability.

% We did not opt for full random generation of this problem
% for pragmatic reasons:
% we want the Nelson-Oppen-style reasoning to be roughly the same for all instances,
% i.e., we want a correct solution to require combined reasoning
% in the theories of arrays, equality, and integers.
% With naive random generation it might well happen that e.g. array operations appear in the formula,
% but no array reasoning is required to prove unsatisfiability.

% For this reason, we prepared a suitable formula to be used as template
For this reason,
we did not opt for full random generation of this problem
but instead prepared a suitable formula to be used as template.
To obtain different problems,
we randomly introduce small perturbations.
These variations may slightly alter the reasoning required inside the theories,
but should not change the fact that reasoning in \emph{all} theories is necessary.

Concretely, we replace each integer-sorted constant symbol $c$ by $c+i$,
where $i \in \{-3,-2,\dots,3\}$ is chosen randomly.
Next, we flatten nested arithmetic terms such as $(c+i)+j$ to $c+k$,
where $i,j,k$ are integers and $k = i+j$.
Since our template contains two constants of integer sort,
this method gives us at most 49 different problems.

% These changes may slightly alter the reasoning required inside the
% theories, but do not change the fact that reasoning in \emph{all} theories
% is necessary.
