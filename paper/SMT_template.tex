We considered first-order formula templates in the combined,
quantifier-free 
theories of equality, arrays and linear integer arithmetic,
corresponding to the logic AUFLIA of
SMT-LIB~\cite{barrett2017smtlib}. We aimed at generating SMT formulas
over which reasoning in all three theories was needed, by exploiting
the DPLL(T) framework~\cite{Tinelli02} in combination with the
Nelson-Oppen decision procedure~\cite{Nelson79}.
%
%One of our goals for this problem was to ensure that a correct solution
%requires combined reasoning in the theories of arrays, equality, and integers.

With a  naive random generation,  it might however happen
that, for example, array reasoning is actually not needed to derive
(un)satisfi\-ability of the generated SMT formula. We therefore set an
SMT formula template and 
%
% We did not opt for full random generation of this problem
% for pragmatic reasons:
% we want the Nelson-Oppen-style reasoning to be roughly the same for all instances,
% i.e., we want a correct solution to require combined reasoning
% in the theories of arrays, equality, and integers.
% With naive random generation it might well happen that e.g. array operations appear in the formula,
% but no array reasoning is required to prove unsatisfiability.
%
% For this reason, we prepared a suitable formula to be used as template
%For this reason,
%we did not opt for full random generation of this problem
%but instead prepared a suitable formula to be used as template.
%To obtain different problems,
%we
randomly introduced small perturbations in this template,  so that the
theory-specific reasoning in all generated SMT instances is
different while reasoning in all theories is necessary.
%that the resulting SMT instances .
%These variations may slightly alter the reasoning required inside the theories,
%but should not change the fact that reasoning in \emph{all} theories is necessary.
For doing so, we considered an SMT template with two constants of
integer sorts and replaced an integer-sorted constant symbol $c$ by  integer-sorted terms $c+i$,
where $i \in \{-3,-2,\dots,3\}$ is chosen randomly. We flattened nested arithmetic terms such as $(c+i)+j$ to $c+k$,
where $i,j,k$ are integers and $k = i+j$. As a result, we
generated 49 different SMT problems; we illustrate one such formula, together
with its reasoning tasks, in Problem~2 of Appendix~\ref{appendixA}.

% These changes may slightly alter the reasoning required inside the
% theories, but do not change the fact that reasoning in \emph{all} theories
% is necessary.
