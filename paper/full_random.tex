We first describe our solution for generating automated reasoning
benchmarks in a fully automated and random manner. We used this
setting for generating exam problems on SAT solving and first-order
theorem proving, by 
% we discuss the two remaining problems of our exam sheet.
%For these problems, we generate instances randomly,
filtering out exam problem instances that are either too
hard or too easy. Throughout this paper, we assume basic familiarity with standard first-order
logic and refer to the literature~\cite{SAT09,Vampire13} for further details.


\subsection{Boolean Satisfiability (SAT)}\label{sec:sat}

% We describe our random generation of exam problems on SAT solving by
% first
% showing  one generated SAT exam problem, as
% below.
In our exam problem on SAT solving, students were asked to
(a) determine which atoms are of pure polarity in the formula,
(b) compute a polarity-optimized clausal normal form~\cite{Tseytin70},
and (c) decide satisfiability of the computed CNF formula by applying the DPLL algorithm.
% Thus, our task for this problem was to generate suitable propositional formulas.

Randomly generating propositional formulas in a naive setting would lead
to a huge variety of formulas,
spanning both formulas for which the above questions on SAT solving
are trivial to answer (e.g. clauses as propositional tautologies)
and others requiring much more effort
(e.g. arbitrary formulas using only $\leftrightarrow$).
More work was thus needed to
ensure comparable workload for solving exam sheets.
%This is obviously undesirable in an exam setting,
%where the tasks should ultimately be challenging, but still solvable by hand.
Furthermore,
the variation in difficulty between different exams should be kept as small as possible,
to make the setting as fair to the examinees as possible.

To this end,
we identified several  syntactical characteristics that the exam problems on SAT solving should exhibit,
and filtered the generated formulas by these, as summarized  partially
below. \smallskip
%
% The criteria are the following:
% Size: 7 connectives
% [ numAtoms fm == 3
% , -- there is at least one atom with pure polarity
%   hasAtomPolarity Neg fm || hasAtomPolarity Pos fm
% , not (anySubformula isNestedNot fm)
% , -- at least one but at most two equivalences
%   let n = countSubformulas isIff fm
%   in 1 <= n && n <= 2
% , anySubformula isImp fm
% , anySubformula isNot fm
% , anySubformula isAnd fm || anySubformula isOr fm
% , not (anySubformula isTrivialBinaryConnective fm)
% , hasVarietyInDefinitionalNF fm
% , length (models fm) <= 6
% , nestedLatexParens fm < 3
% ]

%\begin{compactenum}
\noindent(1) %    \item
        The SAT formula contains exactly seven logical connectives 
        and exactly three different propositional variables. \smallskip
        % (but each of these may appear multiple times).
    % \item
    %     The formula contains exactly seven connectives.
    % \item
    %     The formula contains exactly three different atomic propositions
    %     (but each of these may appear multiple times).

\noindent(2)        
%    \item\label{item:polarity}
        There is at least one atom that appears with a pure polarity.\smallskip
        % (i.e., either only in positive position or only in negative
        % position).
        
\noindent(3) %   \item
        The connectives ``$\leftrightarrow$'', ``$\rightarrow$'', and ``$\lnot$'' appear at least once,
        with ``$\leftrightarrow$'' appearing at most twice.
        At least one of ``$\land$'' and ``$\lor$'' appears. \smallskip
    % \item
    %     The connective ``$\leftrightarrow$'' appears at least once but at most twice.
    % \item
    %     The connectives ``$\rightarrow$'' and ``$\lnot$'' appear at least once.
    % \item
    %     At least one of the connectives ``$\land$'' or ``$\lor$''
    %     appears in the formula.
        
%LK commented out these 2
    %   \item
%        There is no subformula for the form $\lnot \lnot \varphi$ for any formula $\varphi$.
%    \item
 %       There are no trivial binary subformulas such as $p \land
 %       \lnot p$ or $p \lor p$.
 %       LK end
        
    % \item
    %     If a binary connective has a literal as argument,
    %     the other argument cannot also be a literal containing the same atomic proposition.
    %     % No binary connective has two literals that contain the same atomic proposition.
    %     % There is no binary connective that
    %     For example, this excludes subformulas such as $p \land \lnot p$ and $p \lor p$,
    %     but not $p \rightarrow q$.

\noindent(4)
        The polarity-optimized clausal normal form results in a set of definitions,
        each of which is of the form
        $n \circ \varphi$ with $\circ \in \{ \rightarrow, \leftarrow, \leftrightarrow \}$,
        a fresh propositional variable~$n$, and a formula~$\varphi$.
        We restrict the SAT formula such that at least two of the choices for
        $\circ$ appear in the clausal normal form (CNF). \smallskip

\noindent(5)
        The SAT formula has at most $6$ models.
        % Since there are $2^3 = 8$ different interpretations,
        % this means the formula is not valid.
        \smallskip

% \noindent(6)
%         To avoid accidental difficulty introduced by visual complexity,
%         the \LaTeX{} rendering of the formula should have a parenthesis nesting level of at most two.\smallskip
%\end{compactenum}

While the combination of the above conditions might seem very restrictive,
we note that 
%Thus, we enumerated the sample space to see how much variety is possible:
%as it turns out,
there are 20\,390\,076 different SAT formulas satisfying the above
criteria.
% Further,
% if we do not want to distinguish formulas that differ only by a permutation of atoms,
% 3\,398\,346 formulas remain.
% Thus, we are able to generate a large
% number of unique SAT formulas to be used in online examinations and
% beyond.





\subsection{Non-Ground Superposition with Redundancy}\label{sec:fo}

Moving beyond boolean satisfiability, we developed a random problem
generator for first-order formulas with equality, in the setting of
superposition-based first-order theorem proving with redundancy elimination~\cite{Rubio01,Vampire13}.
In this problem, a concrete inference\footnote{i.e., an instance of an inference rule as opposed to the rule itself}
was given to the students, and their task was to
(a) prove that the inference is sound
and (b) that the inference is a simplification inference.
% We illustrate the first-order reasoning task of our
% exam in our randomly generated Example~\ref{ex:fo}.

% For generating first-order reasoning problems similar to
% Example~\ref{ex:fo}, we randomized problem generation in the setting
% of simplification inferences within saturation-based theorem
% proving.
We recall that a simplification inference is an inference
that removes clauses from the proof search space, whereas a generating
inferences add new clauses to the search space~\cite{Vampire13}. In our work, we
considered the simplification inference of 
\emph{subsumption resolution} 
\begin{equation}\label{eq:sr}
    \infer[]{
      D
      }{
      A\vee C
      &
      {\neg B \vee D}
    }
    \qquad \text{or}\qquad
  %
    \infer[]{
      D
      }{
      \neg A\vee C
      &
      {B \vee D}
    }
      \end{equation}
      where $A,B$ are atoms and $C,D$ are clauses
      such that for some substitution (or mgu) $\theta$ we have $A\theta\vee
C\theta\subseteq B\vee D$, and hence the second premise $\neg B \vee
D$ (or $B \vee
D$) of~\eqref{eq:sr} is redundant and can  be deleted from the search
space after applying~\eqref{eq:sr} within proof search.
%
We randomly generated first-order instances of the inference rule~\eqref{eq:sr}, as
discussed next. Our setting could however be easily extended to other
simplification inferences such as subsumption demodulation~\cite{Rath20},
and even generating inferences.

\noindent (1) To randomly generate first-order terms and literals,
we fix  a first-order signature consisting of % three sets specifying the allowed
predicate and function symbols and specify a set of logical
variables. 
%variables.  % (with arity),
%function symbols, % (with arity),
%and variables.
We control the shape of the generated terms
by giving bounds on the \emph{depth} of the term,
that is the maximal nesting level of function calls
(e.g., a constant symbol $b$ has depth 0, while the term $g(f(x),d)$ has depth 2).\smallskip

% The extension to generate random inferences is now straightforward.
% First, we generate the second premise, which should always be non-ground.
% To this end, we invoke our literal generator twice, filtering out any ground literals.

\noindent (2)  To generate random instances of~\eqref{eq:sr}, 
we first generate non-ground clauses $C_2 \coloneqq L_1 \lor L_2$
% corresponding to an instance of the second premise of~\eqref{eq:sr}.
corresponding to an instance of the first premise of~\eqref{eq:sr}.
To this end, we generate a random uninterpreted literal $L_1$ containing exactly one variable occurrence,
and a random equality literal $L_2$ containing at least two occurrences of a different variable.\smallskip
%The restrictions on variable occurrences are implemented as
%post-filters, ensuring that the (i) the clause $C_2$ is non-ground
%and (ii) finding the mgu $\theta$ is of similar difficulty for all examinees.\medskip

% Following this, we generate another uninterpreted literal $L_3$.
% Here, we also check that at least one function symbol of arity 2 appears in at least one of the literals.

\noindent (3)  We next generate the clause
$C_1 \coloneqq \overline{L_1\theta} \lor L_2\theta \lor L_3$
as an instance of the second premise of~\eqref{eq:sr}
where $\theta$ is a randomly generated grounding substitution,
$L_3$ is a randomly generated ground literal,
and
$\overline{L}$ is the complementary%
% \footnote{i.e., $\overline{L} = \lnot L$ and $\overline{\lnot L} = L$.}
literal to $L$.\smallskip
%Note that is very easy to restrict the term generation to ground terms:
%we simply fix the set of variables in the desired signature to the
%empty set before calling the generator.

\noindent (4)
We set $C_3 \coloneqq L_3 \lor L_2\theta$ as an instance of the
conclusion of~\eqref{eq:sr}, yielding thus the inference $\infer[]{
      C_3
      }{
      C_1
      &
      {C_2}}$ as an instance of~\eqref{eq:sr}.\smallskip

 

% Following this, we generate a ground substitution $\theta$ by randomly generating two ground terms.
% Note that is very easy to restrict the term generation to ground terms:
% we simply fix the set of variables in the desired signature to the empty set before calling the generator.
% The first premise is then $C_1 \coloneqq \overline{L_1\theta} \lor L_3 \lor L_2\theta$,
% where $\overline{L}$ is the complementary%
% \footnote{i.e., $\overline{A} = \lnot A$ and $\overline{\lnot A} = A$.}
% literal to $L$.

%At this point, we write the inference to a *.tex-file.
%We also extract the actual signature from the generated inference and output it to a separate *.tex-file.

%\noindent (5) We encode the randomly generated instance of~\eqref{eq:sr} 
%in the SMT-LIB syntax~\cite{barrett2017smtlib} in order to perform
%sanity checks of 
%soundness ($C_1, C_2 \models C_3$)
%and redundancy 
%($C_3, C_2 \models C_1$), ensuring that the randomly
%generated instance of~\eqref{eq:sr} is indeed a simplification
%inference. For these sanity checks, we used the Vampire theorem
%prover~\cite{Vampire13}.\smallskip

We found that with the concrete signature used for our exam,
the discussed method can generate over $6.5 \cdot 10^9$ different instances
of this inference.\todo{add final number once it has completed}
