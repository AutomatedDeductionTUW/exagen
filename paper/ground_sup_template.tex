For generating quantifier-free first-order formulas with equalities,
over which ground and ordered superposition reasoning had to be employed, we
aimed at (i)  generating  unsatisfiable sets $S$ of ground formulas with
uninterpreted functions symbols, such that (ii)  refutation proofs
of $S$ had similar lengths and complexities. Similarly to
Section~\ref{sec:smt}, we fixed a template for $S$  and only varied
the KBO simplification ordering $\prec$ to be used for refuting  $S$
within the superposition calculus. To this end, we considered
variations of weight functions $w$ and symbol precedence $\gg$ over $S$,
yielding thus different KBO $\succ$ to be used for refuting $S$.
The main steps of our approach are summarized below.\smallskip

\noindent(1) We fixed the template for $S$ to be the following set of four clauses
\begin{align}
  &E(F(X)) = a \,\lor\, E(G(Y)) = a \label{gs:1} \\
  &F(X) = a \,[\, \lor\, H(b) \not= H(b) \,] \label{gs:2} \\
  &G(Y) = a \,[\, \lor\, H(b) \not= H(b) \,] \label{gs:3} \\
  &E(a) \not= a \,[\, \lor\, H(b) \not= H(b) \,] \label{gs:4},
\end{align}
where $E, F, G, H \in \{f, g\}$ and $X, Y \in \{a, b\}$.\smallskip


\noindent(2)
We created
instances of $S$ of this template ensuring that no clause in $S$ is
redundant, by considering the following constraints.
\begin{itemize}
  \item $E \not = H$ and $F(X) \not = G(Y)$;
  \item Either $X$ or $Y$ is not $a$. Similarly, either $F$ or $G$ is not $E$;
  \item The literal $H(b) \not = H(b)$ is in exactly one of the
    clauses~\eqref{gs:2},~\eqref{gs:3},~\eqref{gs:4}.
\end{itemize}
%These properties ensure that no clause in the set is redundant,
%and that different instances of the set look sufficiently different.
As a result, we derive 12 instances of $S$ satisfying the above properties.\smallskip
%
%An example of such instance of the clause set is:
%\begin{align}
%  &f(g(a)) = a \,\lor\, f(f(b)) = a \label{gs:1} \\
%  &g(a) = a  \label{gs:2} \\
%  &f(b) = a \, \lor\, g(b) \not= g(b) \label{gs:3} \\
%  &f(a) \not= a  \label{gs:4},
%\end{align}

%Generating the weight function and symbol precedence was slightly more
%complicated. The reason was that we wanted

\noindent(3) We considered the term algebras induced by the generated
instances of $S$ and  generated KBO orderings $\prec$ such that
refuting the respective instances of $S$ using $\prec$ requires
ordering terms both using weight $w$ and precedence $\gg$. %To make the exam problem fair and sufficiently challenging, we wanted to find
%weight and precedence such that the resulting ordering of the terms used
%in the refutation would depend not only on the weights, but also on the precedence,
%and
In addition, we imposed that  either $F(X) \prec a \prec G(Y)$ 
or $G(Y) \prec a \prec F(X)$ hold. With such orderings $\prec$,
the shortest refutations of instances of $S$ are of the same length, and in
at least one application of superposition, $a$ is replaced by either $F(X)$ or $G(Y)$
in the resulting clause. % (this is desirable because it contradicts
%an incorrect intuition that a constant should always be smaller than a term
%containing a unary function).
By fulfilling these conditions, we generated eight
different KBOs $\prec$ to be used with instances of $S$, as
illustrated in Table~\ref{tab:ground-sup} of Appendix~\ref{appendixB}. \smallskip

%Then, for each instance of the clause set, we generated three instances of the exam problem
%using three out of the eight possible orderings, selected depending on the values
%of $E, F, G, H, X$ and $Y$.
\noindent(3) As a result, we obtained 36 different problems (combinations of
instances of $S$ and $\prec$) for  the ground
superposition reasoning task of our exam; Problem~3 of
Appendix~\ref{appendixA} shows such an instance.

