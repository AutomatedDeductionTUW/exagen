Another problem for our exam was to refute a set of ground formulas
(containing uninterpreted functions and equality),
using superposition with KBO ordering $\prec$ based on
a given weight function and symbol precedence.
We wanted the refutations of all instances of the problem to be
of similar length and complexity (w.r.t. verifying the conditions
of rules used in the refutation). To this end, we designed a simple
template consisting of four clauses:
\begin{align}
  &E(F(X)) = a \,\lor\, E(G(Y)) = a \label{gs:1} \\
  &F(X) = a \,[\, \lor\, H(b) \not= H(b) \,] \label{gs:2} \\
  &G(Y) = a \,[\, \lor\, H(b) \not= H(b) \,] \label{gs:3} \\
  &E(a) \not= a \,[\, \lor\, H(b) \not= H(b) \,] \label{gs:4},
\end{align}
where $E, F, G, H \in \{f, g\}$ and $X, Y \in \{a, b\}$.
%
Then we created
instances of this template with the following properties:
\begin{itemize}
  \item $E \not = H$,
  \item $F(X) \not = G(Y)$,
  \item either $X$ or $Y$ is not $a$,
  \item either $F$ or $G$ is not $E$,
  \item the $H(b) \not = H(b)$ literal is in exactly one of the
    clauses~\eqref{gs:2},~\eqref{gs:3},~\eqref{gs:4}.
\end{itemize}
These properties ensure that no clause subsumes another clause,
and that different instances of the set look sufficiently different.
There are 12 instances satisfying the above properties.
%
%An example of such instance of the clause set is:
%\begin{align}
%  &f(g(a)) = a \,\lor\, f(f(b)) = a \label{gs:1} \\
%  &g(a) = a  \label{gs:2} \\
%  &f(b) = a \, \lor\, g(b) \not= g(b) \label{gs:3} \\
%  &f(a) \not= a  \label{gs:4},
%\end{align}

Generating the weight function and symbol precedence was slightly more
complicated. The reason was that we wanted the ordering of the terms used
in the refutation to depend not only on the weights, but also on the ordering,
and at the same time we wanted it to be either
the case that $F(X) \prec a \prec G(Y)$, or $G(Y) \prec a \prec F(X)$. With such an ordering,
the refutation of the set is always of the same length, and in
at least one application of superposition, $a$ will be replaced by either $F(X)$ or $G(Y)$
in the resulting clause (this is desirable because it contradicts
an incorrect intuition that a constant should always be smaller than a term
containing a unary function).
To fulfill these conditions, we designed four templates of weights and precedences,
parametrized by a function symbol $I \in \{f, g\}$, yielding a total of eight
different orderings.
Then, for each instance of the clause set, we generated three instances of the exam problem
using three combinations of weight and precedence selected from the eight possible combinations
depending on the values of $E, F, G, H, X$ and $Y$.
Thus, in total we obtained 36 different instances of the ground superposition problem.
The weight and precedence combinations as well as the conditions used to pick
three of them for each clause set are displayed in Table~\ref{tab:ground-sup}.

\todo{Move the table and the following paragraph to appendix?}

The upper part of the table shows all weight and precedence combinations,
denoted as $w_{i, I}, p_{i, I}$ for $i \in \{1, 2, 3, 4\}$
and $I \in \{f, g\}$ (for convenience, the table contains both $w_{i, f}$ and $w_{i, g}$, as well as
$p_{i, f}$ and $p_{i, g}$ for all values of $i$).
The lower part of the table displays the values of $i_1, I_1; i_2, I_2$; and $i_3, I_3$,
corresponding to the three weight and precedence combinations selected for
each instance of the clause set.

\begin{table}
\begin{center}
\begin{tabular}{r@{\hskip 0.5em}c c c c@{\hskip 0.5em} |@{\hskip 0.5em} l@{\hskip 1em} ||@{\hskip 0.5em} r@{\hskip 0.5em} c c c c@{\hskip 0.5em} |@{\hskip 0.5em} l}
  weight of: & $f$ & $g$ & $a$ & $b$ & precedence
  & weight of: & $f$ & $g$ & $a$ & $b$ & precedence \\ \hline
  $w_{1,f}:$  & 1   & 3   & 2   & 1   & $p_{1,f}: a \gg b \gg f \gg g$ &
  $w_{1,g}:$  & 3   & 1   & 2   & 1   & $p_{1,g}: a \gg b \gg g \gg f$ \\ \hline
  $w_{2,f}:$  & 0   & 3   & 2   & 1   & $p_{2,f}: f \gg a \gg g \gg b$ &
  $w_{2,g}:$  & 3   & 0   & 2   & 1   & $p_{2,g}: g \gg a \gg f \gg b$ \\ \hline
  $w_{3,f}:$  & 0   & 1   & 3   & 1   & $p_{3,f}: f \gg a \gg b \gg g$ &
  $w_{3,g}:$  & 1   & 0   & 3   & 1   & $p_{3,g}: g \gg a \gg b \gg f$ \\ \hline
  $w_{4,f}:$  & 1   & 2   & 3   & 1   & $p_{4,f}: g \gg f \gg a \gg b$ &
  $w_{4,g}:$  & 2   & 1   & 3   & 1   & $p_{4,g}: f \gg g \gg a \gg b$ \\ \hline
  % original table:
  %$w_1(h_1) = w_1(b) = 1, w_1(a) = 2, w_1(h_2) = 3$ & \quad $p_1: a \gg b \gg h_1 \gg h_2$ \\
  %$w_2(h_1) = 0, w_2(b) = 1, w_2(a) = 2, w_2(h_2) = 3$ & \quad $p_2: h_1 \gg a \gg h_2 \gg b$ \\
  %$w_3(h_1) = 0, w_3(b) = w_3(h_2) = 1, w_3(a) = 3$ & \quad $p_3: h_1 \gg a \gg b \gg h_2$ \\
  %$w_4(b) = w_4(h_1) = 1, w_4(h_2) = 2, w_4(a) = 3$ & \quad $p_4: h_2 \gg h_1 \gg a \gg b$ \\
\end{tabular}

\hspace*{0.5em}

\begin{tabular}{l@{\hskip 1.05em} || c | c | c}
\hline
condition & $i_1, I_1$ & $i_2, I_2$ & $i_3, I_3$ \\ \hline
$F \not = G$ and $X \not= Y$ & $1,E$ & $2, E$ & $3, E$ \\
$F \not = G$ and $X = Y$ & $1, H$ & $2, E$ & $4, H$ \\
$F = G$ and $X \not= Y$ & $1, H$ & $2, H$ & $3, E$
\end{tabular}
\hspace*{0.5em}
\caption{Weights and precedences for the ground superposition problem.}
\end{center}
\label{tab:ground-sup}
\end{table}
