For generating quantifier-free first-order formulas with equalities,
over which ground and ordered superposition reasoning had to be employed (Problem 3 of Figure~\ref{fig:exam}), we
aimed at (i)  generating  unsatisfiable sets $S$ of ground formulas with
uninterpreted functions symbols, such that (ii)  refutation proofs
of $S$ had similar lengths and complexities. Similarly to
Section~\ref{sec:smt}, we fixed a template for $S$  and only varied
its instantiation and
the Knuth-Bendix ordering (KBO)~\cite{Knuth1970}~$\succ$ to be used for refuting  $S$
within the superposition calculus. To this end, we considered
variations of weight function $w$ and symbol precedence $\gg$ over $S$,
yielding thus different KBOs $\succ$ to be used for refuting $S$.
The main steps of our approach are summarized below.\smallskip


\begin{table}[t]
\centering
\begin{tabular}{r@{\hskip 0.4em}c c c c@{\hskip 0.4em} |@{\hskip 0.4em} l@{\hskip 0.8em} ||@{\hskip 0.4em} r@{\hskip 0.4em} c c c c@{\hskip 0.4em} |@{\hskip 0.4em} l}
  weight of: & $f$ & $g$ & $a$ & $b$ & precedence
  & weight of: & $f$ & $g$ & $a$ & $b$ & precedence \\ \hline
  $w_{1,f}:$  & 1   & 3   & 2   & 1   & $p_{1,f}: a \gg b \gg f \gg g$ &
  $w_{1,g}:$  & 3   & 1   & 2   & 1   & $p_{1,g}: a \gg b \gg g \gg f$ \\ \hline
  $w_{2,f}:$  & 0   & 3   & 2   & 1   & $p_{2,f}: f \gg a \gg g \gg b$ &
  $w_{2,g}:$  & 3   & 0   & 2   & 1   & $p_{2,g}: g \gg a \gg f \gg b$ \\ \hline
  $w_{3,f}:$  & 0   & 1   & 3   & 1   & $p_{3,f}: f \gg a \gg b \gg g$ &
  $w_{3,g}:$  & 1   & 0   & 3   & 1   & $p_{3,g}: g \gg a \gg b \gg f$ \\ \hline
  $w_{4,f}:$  & 1   & 2   & 3   & 1   & $p_{4,f}: g \gg f \gg a \gg b$ &
  $w_{4,g}:$  & 2   & 1   & 3   & 1   & $p_{4,g}: f \gg g \gg a \gg b$ 
  % original table:
  %$w_1(h_1) = w_1(b) = 1, w_1(a) = 2, w_1(h_2) = 3$ & \quad $p_1: a \gg b \gg h_1 \gg h_2$ \\
  %$w_2(h_1) = 0, w_2(b) = 1, w_2(a) = 2, w_2(h_2) = 3$ & \quad $p_2: h_1 \gg a \gg h_2 \gg b$ \\
  %$w_3(h_1) = 0, w_3(b) = w_3(h_2) = 1, w_3(a) = 3$ & \quad $p_3: h_1 \gg a \gg b \gg h_2$ \\
  %$w_4(b) = w_4(h_1) = 1, w_4(h_2) = 2, w_4(a) = 3$ & \quad $p_4: h_2 \gg h_1 \gg a \gg b$ \\
\end{tabular}
\vspace*{0.3em}
\caption{Weights and precedences for the ground superposition problem.}
\label{tab:ground-sup-wp}
\vspace*{-2em}
\end{table}

\begin{itemize}
\item[(i)] We fixed the template for $S$ to be the following set of four clauses
\begin{align}
  &E(F(X)) = a \,\lor\, E(G(Y)) = a \label{gs:1} \\
  &F(X) = a \,[\, \lor\, H(b) \not= H(b) \,] \label{gs:2} \\
  &G(Y) = a \,[\, \lor\, H(b) \not= H(b) \,] \label{gs:3} \\
  &E(a) \not= a \,[\, \lor\, H(b) \not= H(b) \,] \label{gs:4},
\end{align}
where $E, F, G, H \in \{f, g\}$, $X, Y \in \{a, b\}$, and the literal
in $[~]$ is added to the clauses optionally.\\


\item[(ii)]
We created
instances of $S$ of this template ensuring that no clause in $S$ is
redundant, by considering the following constraints.
\begin{itemize}
  \item $E \not = H$ and $F(X) \not = G(Y)$;
  \item Either $X$ or $Y$ is not $a$. Similarly, either $F$ or $G$ is not $E$;
  \item The literal $H(b) \not = H(b)$ is in exactly one of the
    clauses~\eqref{gs:2},~\eqref{gs:3},~\eqref{gs:4}.
\end{itemize}
%These properties ensure that no clause in the set is redundant,
%and that different instances of the set look sufficiently different.
As a result, we produced 12 instances of $S$ satisfying the above properties.\\
%
%An example of such instance of the clause set is:
%\begin{align}
%  &f(g(a)) = a \,\lor\, f(f(b)) = a \label{gs:1} \\
%  &g(a) = a  \label{gs:2} \\
%  &f(b) = a \, \lor\, g(b) \not= g(b) \label{gs:3} \\
%  &f(a) \not= a  \label{gs:4},
%\end{align}

%Generating the weight function and symbol precedence was slightly more
%complicated. The reason was that we wanted


\begin{table}[t]
\begin{center}
\begin{tabular}{l@{\hskip 1.05em} || c | c | c}
condition & $i_1, I_1$ & $i_2, I_2$ & $i_3, I_3$ \\ \hline
$F \not = G$ and $X \not= Y$ & $1,E$ & $2, E$ & $3, E$ \\
$F \not = G$ and $X = Y$ & $1, H$ & $2, E$ & $4, H$ \\
$F = G$ and $X \not= Y$ & $1, H$ & $2, H$ & $3, E$
\end{tabular}
\vspace*{0.4em}
\caption{Assignment of KBOs to instances of the ground superposition problem.}
\label{tab:ground-sup-as}
\end{center}
\end{table}

\item[(iii)] We considered the term algebras induced by the generated
instances of $S$ and designed KBOs $\succ$ such that
refuting the respective instances of $S$ using $\succ$ requires
ordering terms both using weight $w$ and precedence $\gg$. %To make the exam problem fair and sufficiently challenging, we wanted to find
%weight and precedence such that the resulting ordering of the terms used
%in the refutation would depend not only on the weights, but also on the precedence,
%and
In addition, we imposed that  either $F(X) \succ a \succ G(Y)$ 
or $G(Y) \succ a \succ F(X)$ holds. With such orderings $\succ$,
the shortest refutations of instances of $S$ are of the same length, and in
at least one application of superposition, $a$ is replaced by either $F(X)$ or $G(Y)$
in the resulting clause. % (this is desirable because it contradicts
%an incorrect intuition that a constant should always be smaller than a term
%containing a unary function).
We generated eight different KBOs $\succ$ fulfilling these conditions.
The weights and precedences used to generate the KBOs
are displayed in Table~\ref{tab:ground-sup-wp}.
The table shows all weight and precedence combinations,
denoted as $w_{i, I}, p_{i, I}$ for $i \in \{1, 2, 3, 4\}$
and $I \in \{f, g\}$.\footnote{Note that for all values of~$i$,
$w_{i, f}(f) = w_{i, g}(g)$ and $w_{i, f}(g) = w_{i, g}(f)$,
and the precedences $p_{i, f}, p_{i, g}$ are the same except for
the precedence of $f, g$. However,
for convenience, the table contains both $w_{i, f}$
and $w_{i, g}$, as well as $p_{i, f}$ and $p_{i, g}$ for all values of~$i$.}
Each instance of $S$ was combined with three different KBOs, generated
by pairs
$(w_{i_1, I_1}, p_{i_1, I_1}), (w_{i_2, I_2}, p_{i_2, I_2}), (w_{i_3, I_3}, p_{i_3, I_3})$.
The values of $i_1, I_1, i_2, I_2, i_3, I_3$ are chosen
based on the values of $F, G, X, Y$, as expressed by the
conditions in Table~\ref{tab:ground-sup-as}.
\smallskip
\end{itemize}
%Then, for each instance of the clause set, we generated three instances of the exam problem
%using three out of the eight possible orderings, selected depending on the values
%of $E, F, G, H, X$ and $Y$.

  Ultimately, we obtained 36 different problems (combinations of
instances of $S$ and $\succ$) for  the ground
superposition reasoning task of our exam.  Problem~3 of
Figure~\ref{fig:exam} shows such an instance.
