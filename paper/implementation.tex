Various programs/scripts tied together by a control script.
Running the control script generates all problems and puts them together, resulting in $n$ different exams in PDF format.


Latex template:
leave holes for formulas/signatures that will be inserted using the command ``\textbackslash{}input''.


Problem templates: (todo: describe impl of smt and groundsup)


For the full random generation of the SAT and redundancy problems,
we were initially inspired by Haskell's QuickCheck library.
\todo{cite?
Claessen, Koen and Hughes, John (2000). "QuickCheck: A Lightweight Tool for Random Testing of Haskell Programs" (PDF). Proceedings of the International Conference on Functional Programming (ICFP), ACM SIGPLAN.
}
We used QuickCheck for the first prototype, but realized that we need backtracking in the generator.

Added mtl-style typeclass 'MonadChoose' with a primitive operation 'choose' for choosing an element from a list of choices.
Our generator implementations are generic over the monad, constrained by this typeclass.

We used two concrete implementations to evaluate generators:
1. 'RandomChoiceT', a monad transformer that implements 'choose' as uniform random choice with backtracking support
    (can be thought of as 'ListT' where 'choose' is like the normal monadic bind but shuffles the list with a random permutation first).
    This is what we actually use to generate random exams.
2. a standard list monad (or similar) to enumerate the sample space.
    This second evaluation method lets us ensure that the sample space is sufficiently large.

% Thus, re-implemented nondeterministic monad with an additional primitive operation for uniform random choice of an element from a list.
% We used mtl-style typeclass for the generator implementations;
% this allows us to use the same generator in two ways.
% 1. get a random element (by evaluating in a 
% 2. list all possible elements (by evaluating in the built-in list monad, or a similar one)
% The second evaluation method lets us ensure that the sample space is sufficiently large.


