Our implementation consists of various programs that are tied together by a control script.
The control script calls all problem generators
and compiles the exam sheets, resulting in $n$ different exams in PDF format.

The exam sheet template is a \LaTeX{} file
where the concrete formulas and signatures
have been replaced by ``\textbackslash{}input'' commands.

The ground superposition problem generator was implemented in Python,
and utilizes a simple iteration over cartesian product of symbols
with filtering.

The SAT, SMT and redundancy problem generators have been implemented in Haskell.
The implementation of formulas and terms as inductive data types
in combination with pattern matching has been very convenient to work with.
An interesting aspect of our implementation is that
our generators are generic over the evaluation method of choice points.
This allowed us to use the same code both for random sampling (to generate exams)
and for enumerating the sample space (to check whether it is sufficiently large).
