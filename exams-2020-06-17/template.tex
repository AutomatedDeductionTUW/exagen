
\def\Drop{{\sf DROP}}

\def\z{{\tt 0}}
\def\o{{\tt 1}}
\def\t{{\tt 2}}

\def\qu{\quad}
\def\qqu{\qu\qu}

\def\set#1{\{#1\}}
\def\ra{\rightarrow}
\def\sseq{\subseteq}

\def\header{{\vspace*{-7em}
\noindent
Automated Deduction -- SS 2020\\[2pt]
{\Large\bf Final Exam {\small -- June 17, 2020}}\\
\hrule\medskip}}

\documentclass{article}[8pt]
\usepackage{times,fullpage}
\usepackage{epic}
\usepackage{eepic}
\usepackage{latexsym}

%
% Logic
%
\newcommand{\andl}{\wedge}
\newcommand{\bigandl}{\bigwedge}
\newcommand{\orl}{\vee}
\newcommand{\bigorl}{\bigvee}
\newcommand{\notl}{\neg}
\newcommand{\impl}{\rightarrow}
\newcommand{\imply}{\rightarrow}
\newcommand{\implies}{\rightarrow}
\newcommand{\iffl}{\leftrightarrow}
\newcommand{\true}{1}
\newcommand{\false}{0}
\newcommand{\intI}{I}
\newcommand{\complem}[1]{\overline{#1}}
\newcommand{\setof}[1]{\{#1\}}



%\usepackage[top=4pt ,foot=0pt, bottom=0.01in, left=1in, right=.7in]{geometry}

\begin{document}
\pagestyle{empty}
\header

%\noindent
%This exam sheet consist of six problems, yielding a total of 100 points. Good luck!\medskip

%Splitting+ dpll

\medskip

\noindent
{\bf Problem 1.} (25 points)
Consider the formula :
\[ \ldots\]
\begin{itemize}
\item[(a)]
Which atoms are pure in the above formula?
\item[(b)] Compute a clausal normal form $C$ of the above formula by
  applying the CNF transformation algorithm with naming and
  optimization based on polarities of subformulas;
\item[(c)] Decide the satisfiability of the computed CNF formula $C$
  by applying  the DPLL method to $C$. If $C$ is
satisfiable,  give an interpretation which
satisfies it.
\end{itemize}

\medskip
% SMTin theories

\noindent
{\bf Problem 2.} (25 points) %SAT(5)+Separate reasoning(5)+Dec.procedures(5)+model(5)
Consider the formula:
\[\ldots,\]
where $\ldots$  are constants,
$\ldots$ is a unary function symbols,
$A$ is an array constant  and
$+, 1$ are interpreted
in the standard way over the integers. \medskip


\noindent
Use the Nelson-Oppen decision procedure in conjunction with DPLL-based
reasoning in the combination of the theories of arrays,
uninterpreted functions, and linear integer arithmetic.
Use the decision procedures for the theory of arrays and the theory of uninterpreted functions and use
simple mathematical reasoning for deriving new equalities among the
constants in the theory of linear integer arithmetic.
If the formula is satisfiable, give an interpretation that satisfies the formula.\bigskip

%ground superposition
\noindent
{\bf Problem 3.} (25 points)
Consider the KBO ordering $\succ$ generated by the precedence
$ ... $ and the weight function $w$ with $... $.
Let $\sigma$ be a well-behaved selection function
wrt $\succ$.  Consider the set $S$ of ground formulas:
\[
  \begin{array}{l}
...
\end{array}
\]
\noindent Show that $S$ is unsatisfiable by applying saturation on $S$ using an
inference process based on the ground superposition
calculus $\textrm{Sup}_{\succ,\sigma}$ (with the inference rules of
binary  resolution $\textrm{BR}_{\sigma}$ included).
 Give details on what literals
 are selected and which terms are maximal. \bigskip

 % different proof when w(f)=0, w(a)=w(b)=w(c)=1, and f>>a>>b>>c
 % as in this new case f(b)>a


%non-ground superposition, mgu
\noindent{\bf Problem 4.} (25 points)
Consider the following inference:
\[
\begin{array}{l}
 ... \end{array}
\]
in the non-ground
superposition inference system $\textrm{Sup}$ (including the rules of the non-ground
binary resolution  inference system $\textrm{BR}$), where
where $f$ is a function symbol, %$p$ is a predicate symbol, $f$ is a function symbol,
$x$ is a variable and $a,b,c$ are
constants.  %\\
%

          \begin{itemize}
            \item[(a)] Prove that the above inference is a sound
              inference of $\textrm{Sup}$.
            \item[(b)] Is the above inference a simplifying inference
              of $\textrm{Sup}$?
              Justify your answer.
\end{itemize}


\end{document}
