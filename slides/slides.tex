%%%%%%%%%%%%%%%%%%%%%%%%%%%%%%%%%%%%%%%%%
% Beamer Presentation
% LaTeX Template
% Version 1.0 (10/11/12)
%
% This template has been downloaded from:
% http://www.LaTeXTemplates.com
%
% License:
% CC BY-NC-SA 3.0 (http://creativecommons.org/licenses/by-nc-sa/3.0/)
%
%%%%%%%%%%%%%%%%%%%%%%%%%%%%%%%%%%%%%%%%%

%----------------------------------------------------------------------------------------
%	PACKAGES AND THEMES
%----------------------------------------------------------------------------------------

\documentclass[xcolor={table}]{beamer}

\mode<presentation> {

% The Beamer class comes with a number of default slide themes
% which change the colors and layouts of slides. Below this is a list
% of all the themes, uncomment each in turn to see what they look like.

\usetheme{default}
%\usetheme{AnnArbor}
%\usetheme{Antibes}
%\usetheme{Bergen}
%\usetheme{Berkeley}
%\usetheme{Berlin}
%\usetheme{Boadilla}
%\usetheme{CambridgeUS}
%\usetheme{Copenhagen}
%\usetheme{Darmstadt}
%\usetheme{Dresden}
%\usetheme{Frankfurt}
%\usetheme{Goettingen}
%\usetheme{Hannover}
%\usetheme{Ilmenau}
%\usetheme{JuanLesPins}
%\usetheme{Luebeck}
%\usetheme{Madrid}
%\usetheme{Malmoe}
%\usetheme{Marburg}
%\usetheme{Montpellier}
%\usetheme{PaloAlto}
%\usetheme{Pittsburgh}
%\usetheme{Rochester}
%\usetheme{Singapore}
%\usetheme{Szeged}
%\usetheme{Warsaw}

% As well as themes, the Beamer class has a number of color themes
% for any slide theme. Uncomment each of these in turn to see how it
% changes the colors of your current slide theme.

%\usecolortheme{albatross}
%\usecolortheme{beaver}
%\usecolortheme{beetle}
%\usecolortheme{crane}
%\usecolortheme{dolphin}
%\usecolortheme{dove}
%\usecolortheme{fly}
%\usecolortheme{lily}
%\usecolortheme{orchid}
%\usecolortheme{rose}
%\usecolortheme{seagull}
%\usecolortheme{seahorse}
%\usecolortheme{whale}
%\usecolortheme{wolverine}

%\setbeamertemplate{footline} % To remove the footer line in all slides uncomment this line
%\setbeamertemplate{footline}[frame number] % To replace the footer line in all slides with a simple slide count uncomment this line
\setbeamertemplate{footline}{
  \hspace*{0.8em}\insertshorttitle
  \hfill
  \insertframenumber{} \hspace*{0.8em}
  \vspace*{0.3em}
}

\setbeamertemplate{navigation symbols}{} % To remove the navigation symbols from the bottom of all slides uncomment this line
}

\usepackage[table]{xcolor}
%\usepackage{graphicx} % Allows including images
%\usepackage{booktabs} % Allows the use of \toprule, \midrule and \bottomrule in tables
\usepackage{amsmath}
%\usepackage{proof}
%\usepackage{array}
%\usepackage{rotating}
%\usepackage{adjustbox}
%\usepackage{multirow}
\usepackage[T1]{fontenc}

\renewcommand{\implies}{\rightarrow}
\newcommand\Wider[2][3em]{%
\makebox[\linewidth][c]{%
  \begin{minipage}{\dimexpr\textwidth+#1\relax}
  \raggedright#2
  \end{minipage}%
  }%
}

%----------------------------------------------------------------------------------------
%	TITLE PAGE
%----------------------------------------------------------------------------------------

\title{Automated Generation of Exam Sheets for Automated Deduction} % The short title appears at the bottom of every slide, the full title is only on the title page

\author{Petra Hozzov\'a, Laura Kov\'acs, \underline{Jakob Rath}} % Your name
\institute{
Vienna University of Technology%\\ % Your institution for the title page
%\medskip
%\textit{john@smith.com} % Your email address
}
\date{\vspace*{-3em}\small July 29, 2021} % Date, can be changed to a custom date

\begin{document}

\begin{frame}[plain]
\titlepage % Print the title page as the first slide
\end{frame}
\addtocounter{framenumber}{-1}

%----------------------------------------------------------------------------------------
%	PRESENTATION SLIDES
%----------------------------------------------------------------------------------------

%------------------------------------------------

\begin{frame}
\frametitle{Motivation}
\begin{itemize}
\item Remote exams for Automated Deduction course
\item Monitoring students -- not feasible
\item Mulitple-choice exam format -- not sufficient
\end{itemize}
\pause
  \vspace*{1em}
\qquad $\Rightarrow$ generate unique exam for each student
\end{frame}

%------------------------------------------------

%------------------------------------------------

\begin{frame}
\frametitle{Goals}
\begin{itemize}
\item 4 problems requiring complex reasoning:
  SAT solving, SMT solving, ground superposition, first-order theorem proving
\item Tradeoff between problems being sufficiently similar and sufficiently different
\item At least 31 instances of each problem
\item Effcient generation
\item Soundness checking
\end{itemize}
\end{frame}

%------------------------------------------------

%------------------------------------------------

\begin{frame}
\frametitle{Our solution}
Combination of 2 approaches:
\begin{enumerate}
\item Random generation with filtering
\item Variation of templates
\end{enumerate}
\end{frame}

%------------------------------------------------

%------------------------------------------------

\begin{frame}
\frametitle{Random problem generation}
\begin{itemize}
\item Used for problems on SAT solving and first-order theorem proving
\item Generate a lot of problem instances and filter out unsuitable ones
\item TODO: implementation details
\item TODO: examples
\end{itemize}
\end{frame}

%------------------------------------------------

%------------------------------------------------

\begin{frame}
\frametitle{Random variation of problem template}
\begin{itemize}
\item Used for problems on SMT solving and ground superposition
\item Design templates for first-order formulas (and orderings)
  and generate their instances
\item TODO: implementation details
\item TODO: examples
\end{itemize}
\end{frame}

%------------------------------------------------

%------------------------------------------------

\begin{frame}
\frametitle{Implementation summary (TODO better slide title)}
\begin{itemize}
\item Around 2300 lines of code
\item Sanity checks using Z3 and Vampire
\item Conversion to \LaTeX
\end{itemize}
\end{frame}

%------------------------------------------------

%------------------------------------------------

\begin{frame}
\frametitle{Teaching outcomes}
\begin{itemize}
\item Less effort to generate the exams compared to manual method
\item Grading reasonably simple
\item Comparable students' performance
\item High levels of student satisfaction (small sample)
\item Framework reused in Summer 2021
\end{itemize}
\end{frame}

%------------------------------------------------

%------------------------------------------------

\begin{frame}
\frametitle{Conclusions}
TODO some take-away message
\begin{itemize}
\item
\end{itemize}
\end{frame}

%------------------------------------------------


\end{document} 
