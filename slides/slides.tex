%%%%%%%%%%%%%%%%%%%%%%%%%%%%%%%%%%%%%%%%%
% Beamer Presentation
% LaTeX Template
% Version 1.0 (10/11/12)
%
% This template has been downloaded from:
% http://www.LaTeXTemplates.com
%
% License:
% CC BY-NC-SA 3.0 (http://creativecommons.org/licenses/by-nc-sa/3.0/)
%
%%%%%%%%%%%%%%%%%%%%%%%%%%%%%%%%%%%%%%%%%

%----------------------------------------------------------------------------------------
%	PACKAGES AND THEMES
%----------------------------------------------------------------------------------------

\documentclass[xcolor={table}]{beamer}

\mode<presentation> {

% The Beamer class comes with a number of default slide themes
% which change the colors and layouts of slides. Below this is a list
% of all the themes, uncomment each in turn to see what they look like.

\usetheme{default}
%\usetheme{AnnArbor}
%\usetheme{Antibes}
%\usetheme{Bergen}
%\usetheme{Berkeley}
%\usetheme{Berlin}
%\usetheme{Boadilla}
%\usetheme{CambridgeUS}
%\usetheme{Copenhagen}
%\usetheme{Darmstadt}
%\usetheme{Dresden}
%\usetheme{Frankfurt}
%\usetheme{Goettingen}
%\usetheme{Hannover}
%\usetheme{Ilmenau}
%\usetheme{JuanLesPins}
%\usetheme{Luebeck}
%\usetheme{Madrid}
%\usetheme{Malmoe}
%\usetheme{Marburg}
%\usetheme{Montpellier}
%\usetheme{PaloAlto}
%\usetheme{Pittsburgh}
%\usetheme{Rochester}
%\usetheme{Singapore}
%\usetheme{Szeged}
%\usetheme{Warsaw}

% As well as themes, the Beamer class has a number of color themes
% for any slide theme. Uncomment each of these in turn to see how it
% changes the colors of your current slide theme.

%\usecolortheme{albatross}
%\usecolortheme{beaver}
%\usecolortheme{beetle}
%\usecolortheme{crane}
%\usecolortheme{dolphin}
%\usecolortheme{dove}
%\usecolortheme{fly}
%\usecolortheme{lily}
%\usecolortheme{orchid}
\usecolortheme{rose}
%\usecolortheme{seagull}
%\usecolortheme{seahorse}
%\usecolortheme{whale}
%\usecolortheme{wolverine}

%\setbeamertemplate{footline} % To remove the footer line in all slides uncomment this line
%\setbeamertemplate{footline}[frame number] % To replace the footer line in all slides with a simple slide count uncomment this line
\setbeamertemplate{footline}{
  \hspace*{0.8em}\insertshorttitle
  \hfill
  \insertframenumber{} \hspace*{0.8em}
  \vspace*{0.3em}
}

\setbeamertemplate{navigation symbols}{} % To remove the navigation symbols from the bottom of all slides uncomment this line
}

\usepackage[table]{xcolor}
%\usepackage{graphicx} % Allows including images
%\usepackage{booktabs} % Allows the use of \toprule, \midrule and \bottomrule in tables
\usepackage{amsmath}
%\usepackage{proof}
%\usepackage{array}
%\usepackage{rotating}
%\usepackage{adjustbox}
%\usepackage{multirow}
\usepackage{listings}
\usepackage[T1]{fontenc}

\renewcommand{\implies}{\rightarrow}
\newcommand\Wider[2][3em]{%
\makebox[\linewidth][c]{%
  \begin{minipage}{\dimexpr\textwidth+#1\relax}
  \raggedright#2
  \end{minipage}%
  }%
}

\newcommand{\limpl}{\rightarrow}
\newcommand{\liff}{\leftrightarrow}

%----------------------------------------------------------------------------------------
%	TITLE PAGE
%----------------------------------------------------------------------------------------

\title{Automated Generation of Exam Sheets for Automated Deduction} % The short title appears at the bottom of every slide, the full title is only on the title page

\author{
    Petra Hozzov\'a
    \and
    Laura Kov\'acs
    \and
    \texorpdfstring{\underline{Jakob Rath}}{Jakob Rath}
}
\institute{
Vienna University of Technology%\\ % Your institution for the title page
%\medskip
%\textit{john@smith.com} % Your email address
}
\date{\vspace*{-3em}\small July 29, 2021} % Date, can be changed to a custom date

\lstset{
    language=Haskell,
    basicstyle=\small\ttfamily,
    deletekeywords={MonadPlus,filter,length},
    columns=fixed,
}

\begin{document}

\begin{frame}[plain]
\titlepage % Print the title page as the first slide
\end{frame}
\addtocounter{framenumber}{-1}

%----------------------------------------------------------------------------------------
%	PRESENTATION SLIDES
%----------------------------------------------------------------------------------------

%------------------------------------------------

\begin{frame}
\frametitle{Motivation}

    \begin{block}{Automated Deduction}
        \begin{itemize}
            \item Elective Master's level course at TU Wien
            \item SAT solving, SMT solving, first-order ATP
            \item Exercise sheets as homework, open-book final exam
            \item Held remotely in summer semester 2020 (and 2021)
        \end{itemize}
    \end{block}

    % \medskip

    % Challenge:
    % held remotely in summer semester 2020 (and 2021)

    \medskip
    \pause

    \begin{alertblock}{Challenge: remote exams}
        \begin{itemize}
            \item Monitoring students -- not feasible
            \item Multiple-choice exam format -- not sufficient
        \end{itemize}
    \end{alertblock}
    % \begin{itemize}
    %     \item 
    %     \item Remote exams
    %     \item Monitoring students -- not feasible
    %     \item Multiple-choice exam format -- not sufficient
    % \end{itemize}
    \pause
    \vspace*{1em}
    \qquad $\rightsquigarrow$ create a unique exam for each student
\end{frame}

%------------------------------------------------

%------------------------------------------------

\begin{frame}
\frametitle{Goal: unique exam for each student}
\begin{itemize}
\item Four problems requiring complex reasoning:
    \begin{enumerate}
        \item SAT solving
        \item SMT solving
        \item ground superposition
        \item first-order theorem proving
    \end{enumerate}
    \pause
\item Tradeoff:
    \begin{itemize}
    \item Exams should be sufficiently different (prevent cheating)
    \item But also sufficiently similar (fairness)
    \end{itemize}
% \item Tradeoff between exams being sufficiently similar and sufficiently different
    \pause
\item At least 31 instances of each problem
\item Efficient generation
\item Soundness checking
\end{itemize}
\end{frame}

%------------------------------------------------

%------------------------------------------------

\begin{frame}
\frametitle{Our solution}
Combination of 2 approaches:
\begin{enumerate}
\item Random generation with filtering
\item Variation of templates
\end{enumerate}
\end{frame}

%------------------------------------------------

%------------------------------------------------

\begin{frame}
\frametitle{Random problem generation}
\begin{itemize}
\item Used for problems on SAT solving and first-order theorem proving
\item Generate random problem instances
\item Filter out unsuitable ones
% \item TODO: implementation details
% \item TODO: examples
\end{itemize}
\end{frame}

%------------------------------------------------

%------------------------------------------------

\begin{frame}
    \frametitle{Random problem generation -- SAT solving}

    \begin{block}{Exam problem}
        Consider the propositional formula:
        \[
            (r \land \lnot (q \limpl p)) \lor (q \liff \lnot (p \limpl q))
        \]
        \vspace*{-1.5em}
        \begin{enumerate}
            \item Which atoms are pure?
            \item Compute a clausal normal form.
            \item Decide satisfiability using the DPLL algorithm.
        \end{enumerate}
    \end{block}
\end{frame}


\begin{frame}[fragile]
    \frametitle{Random problem generation -- SAT solving}

    \begin{itemize}
        \item
            Implementation in Haskell
        \item
            Inspired by QuickCheck library\footnote{K. Claessen and J. Hughes, ICFP 2000}
        \item
            Specifications are generic over generator monad \texttt{m}
            \\
            (like \texttt{Gen} in QuickCheck, but with backtracking support)

\begin{lstlisting}
class MonadPlus m => MonadChoose m where
  choose :: [a] -> m a
\end{lstlisting}

        \item
            Two concrete implementations:

            \begin{enumerate}
                \item choose by uniform random distribution
                \item enumerate all choices
            \end{enumerate}
    \end{itemize}
\end{frame}


\begin{frame}
    \frametitle{Random problem generation -- SAT solving}

    Constraints during generation
    \begin{itemize}
        \item
            number of connectives
        \item
            no trivial binary connectives such as $p \land \lnot p$
        \item
            ...
    \end{itemize}

    \bigskip
    \pause

    Constraints as post-filters
    \begin{itemize}
        \item
            all three variables appear
        \item
            there is at least one atom of pure polarity
        \item
            variety in definitional NF \\
            (at least two of the forms $n \limpl F$, $F \limpl n$, $n \liff F$ appear)
        \item
            ...
    \end{itemize}
\end{frame}

\begin{frame}
    \frametitle{Random problem generation -- SAT solving}

    Examples:
    \begin{itemize}
        \item $ ( \lnot p \rightarrow \lnot r ) \land ( ( r \land q ) \rightarrow ( p \leftrightarrow q ) ) $
        \item $ ( ( p \rightarrow r ) \rightarrow ( r \leftrightarrow p ) ) \land ( \lnot p \lor ( q \land r ) ) $
        \item $ \lnot ( ( q \lor p ) \land \lnot r ) \rightarrow ( ( r \leftrightarrow p ) \land r ) $
        \item $ ( ( p \land r ) \rightarrow ( r \lor q ) ) \rightarrow ( \lnot r \land ( r \leftrightarrow q ) ) $
        \item $ ( p \lor ( r \land q ) ) \rightarrow \lnot ( \lnot ( p \leftrightarrow q ) \lor q ) $
    \end{itemize}
\end{frame}

%------------------------------------------------

%------------------------------------------------

\begin{frame}
    \frametitle{Random problem generation -- first-order ATP}

    \begin{block}{Exam problem}
        Consider the inference:
        \begin{align*}
            & P(h(g(g(d,d),b))) \lor \lnot P(h(f(d))) \lor f(d) \neq h(g(a,a)) \\
            & \lnot P(h(g(x,b))) \lor f(d) \neq h(g(y,y)) \\
            \cline{1-2}
            & \lnot P(h(f(d))) \lor f(d) \neq h(g(a,a))
        \end{align*}
        \vspace*{-1.5em}
        \begin{enumerate}
            \item Prove that the inference is sound.
            \item Is the inference simplifying?
        \end{enumerate}
    \end{block}
    % TODO: during talk explicitly mention that we have two premises, one conclusion (maybe point at them)
\end{frame}

\begin{frame}
    \frametitle{Random problem generation -- first-order ATP}

    Implementation based on same principles as the SAT problem:
    \begin{enumerate}
        \item Fix a signature
        \item Generate three random literals
        \item Post-filter ensures minimum number of variable occurrences
        \item Generate random substitution
        \item Arrange into inference rule
    \end{enumerate}
\end{frame}

\begin{frame}[fragile]
    \frametitle{Random problem generation -- first-order ATP}

    Excerpt from literal generator:

\begin{lstlisting}
l1 <- mfilter ((==1) . length . variables)
    $ genUninterpretedLiteral sig{ vars = [v1] }
l3 <- genUninterpretedLiteral sig{ vars = [] }
\end{lstlisting}
\end{frame}

%------------------------------------------------

%------------------------------------------------

\begin{frame}
  \frametitle{Random variation of problem template}
\begin{itemize}
\item Used for problems on SMT solving and ground superposition
\item Design templates for first-order formulas (and orderings)
  and generate different instances
\end{itemize}
\end{frame}

%------------------------------------------------

%------------------------------------------------

\begin{frame}
    \frametitle{Random variation of problem template -- SMT solving}

    \begin{block}{Exam problem}
        Is the following formula satisfiable?
        % \begin{align*}
        %     & f(b+1) \neq b+2 \\
        %     \land~
        %     & read(A,f(c+1))=c \\
        %     \land~
        %     & …
        % \end{align*}
        \[
            f(b+1) \neq b+2
            \land
            read(A,f(c+1))=c
            \land
            \dots
        \]
        Use the Nelson-Oppen decision procedure
        in conjunction with DPLL-based reasoning
        in the combination of the theories of
        arrays, uninterpreted functions, and linear arithmetic.
    \end{block}
    % NOTE: why template? naive variations might create problems where theory reasoning is not required.
\end{frame}

\begin{frame}
    \frametitle{Random variation of problem template -- SMT solving}

    Issue with randomly generated formulas: \\
    might not require reasoning in all theories!
    \bigskip
    \pause

    Solution:
    \begin{itemize}
        % \item
            % Naive random generation:
            % Why template?
            % naive random generation might create problems that do
            % not need nelson-oppen style reasoning to prove unsat
            % (but we want the students to use e.g. array reasoning)
        \item
            Provide a formula as template
        \item
            Shift integer constants by a small random value
        \item
            Reasoning task stays essentially the same
            % (modulo some calculations)
        % \item
        %     e.g., replace all occurrences of $b$ by $b+3$.
        % \item TODO: implementation details
        % \item TODO: example
    \end{itemize}

    \begin{example}
        Template:
        \[ f(b) \neq b+1 \]
        Shifting $b$ by $1$:
        \[ f(b+1) \neq b+2 \]
    \end{example}
\end{frame}

%------------------------------------------------

%------------------------------------------------

\begin{frame}
  \frametitle{Random variation of template -- ground superposition}
    \begin{block}{Exam problem}
      Consider the set $S$:
      \vspace*{-0.5em}
      \begin{align*}
        S = \{& \, f(g(b)) = a \lor f(g(a)) = a,
      ~\\~
      & g(b) = a,
      ~\\~
      & g(a) = a,
      ~\\~
        & g(b)  \neq  g(b) \lor f(a)  \neq  a \,\}
      \end{align*}
      Show that $S$ is unsatisfiable by saturation using superposition
      with the Knuth-Bendix ordering generated by:
      \vspace*{-0.5em}
      \begin{gather*}
        f \gg a \gg b \gg g, \text{ and} \\
        w(f) = 0, w(b) = 1, w(g) = 1, w(a) = 3.
      \end{gather*}
    \end{block}

\end{frame}

%------------------------------------------------

%------------------------------------------------

\begin{frame}
  \frametitle{Random variation of template -- ground superposition}

  \begin{enumerate}[<+->]
\item Fix the template for $S$:
{\small
\begin{align*}
  &E(F(X)) = a \,\lor\, E(G(Y)) = a \\
  &F(X) = a \,[\, \lor\, H(b) \not= H(b) \,] \\
  &G(Y) = a \,[\, \lor\, H(b) \not= H(b) \,] \\
  &E(a) \not= a \,[\, \lor\, H(b) \not= H(b) \,],
\end{align*}
}
where $E, F, G, H \in \{f, g\}$, $X, Y \in \{a, b\}$.
%and the literal in $[~]$ is added to the clauses optionally.\\
\item Generate 12 instances of the template without redundancy in $S$
  (using simple syntactic criteria).
%, by considering the following constraints.
%\begin{itemize}
%  \item $E \not = H$ and $F(X) \not = G(Y)$;
%  \item Either $X$ or $Y$ is not $a$. Similarly, either $F$ or $G$ is not $E$;
%  \item The literal $H(b) \not = H(b)$ is in exactly one of the
%    clauses~\eqref{gs:2},~\eqref{gs:3},~\eqref{gs:4}.
%\end{itemize}
%As a result, we produced 12 instances of $S$ satisfying the above properties.\\
\item Design 8 KBOs $\succ$ (3 for each $S$) such that:
  \begin{itemize}
    \item<.-> refuting $S$ using $\succ$ requires ordering using both weight $w$ and precedence $\gg$,
    \item<.-> either $F(X) \succ a \succ G(Y)$ or $G(Y) \succ a \succ F(X)$ holds.
  \end{itemize}
%With such orderings $\succ$,
%the shortest refutations of instances of $S$ are of the same length, and in
%at least one application of superposition, $a$ is replaced by either $F(X)$ or $G(Y)$
%in the resulting clause. (That might be a "non-intuitive" step for some students.)
\end{enumerate}

\end{frame}

%------------------------------------------------

%------------------------------------------------

\begin{frame}
\frametitle{Assembling the parts}
\begin{itemize}
\item Around 2300 lines of code in Haskell and Python
\item Sanity checks using Z3 and Vampire
\item Conversion to \LaTeX
\end{itemize}
\end{frame}

%------------------------------------------------

%------------------------------------------------

\begin{frame}
\frametitle{Teaching outcomes}
\begin{itemize}
\item Less effort to generate the exams compared to manual method
\item Grading reasonably simple
\item Comparable students' performance
\item High levels of student satisfaction (small sample)
\end{itemize}
\end{frame}

%------------------------------------------------

%------------------------------------------------

\begin{frame}
\frametitle{Future plans}
\begin{itemize}
    \item Larger pool of problems
    \item Generate more solutions (where feasible)
    % (future work: generate solutions as well... non-trivial when there are multiple paths)
    \item Instead of random generator, encode constraints into SAT/SMT?
            % (then use the SAT solver for random generation... question is how maintainable such a solution would be...)
\end{itemize}
\end{frame}

%------------------------------------------------

%------------------------------------------------

\begin{frame}
\frametitle{Conclusion}

  \begin{itemize}
    \item Efficient automated generation of exam problems
    \item Balance between variety and fairness
    \item Framework reused in Summer 2021
  \end{itemize}

\bigskip
\pause

\begin{center}
    Thank you!
\end{center}
\end{frame}

%------------------------------------------------


\end{document} 
